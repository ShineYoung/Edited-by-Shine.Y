\documentclass[10pt,a4paper]{article}

\usepackage[UTF8]{ctex}
\usepackage{geometry}
\usepackage{amsmath}
\usepackage{xcolor}
\usepackage{multicol}

\geometry{a4paper,scale=0.8}

\begin{document}

\centerline{\Large{\textbf{2.2 等差数列 (导学案)}}}

% \rightline{备课人:杨习}

%   \section{三维目标}

%   		\subsection{学习目标}
% 			\begin{enumerate}
% 				\item 通过实例,把握等差数列的特点,理解等差数列的概念,能根据定义判断一个数列是等差数列; 
% 				\item 探索并掌握等差数列的通项公式,及通项公式的简单应用; 
% 				\item 通过等差数列概念的归纳概括,培养学生的观察、分析资料的能力,积极思维,追求新知的创新意识. 
% 			\end{enumerate}


  		\subsection{教学重难点}

			\begin{description}

				\item[重点] 理解等差数列的概念, 探索并掌握等差数列的通项公式; 

				\item[难点] 等差数列通项公式推导. 

			\end{description}

		\subsection{学法指导}
			复习——探究——知识应用——巩固练习——小结反思



	\section{教学过程}

		% \subsection{复习旧知} 

		% 	\begin{enumerate}
		% 		\item 数列的概念:
		% 		\item 数列的分类:
		% 				\begin{itemize}
		% 					\item 按项数分:
		% 					\item 按数列项的变化分:
		% 				\end{itemize}
		% 		\item 数列的通项公式:
		% 	\end{enumerate}
		

		\subsection{探究新知} % (fold)
		\label{sub:探究新知}
		
		% subsection 探究新知 (end)
			\subsubsection{等差数列的定义}

			【练习】 判断下列数列中哪些是等差数列,哪些不是?如果是,写出公差$d$, 如果不是,说明理由。 

					\begin{enumerate}
						\item 数列4, 7, 10, 13, 16, $\ldots$; 
						\item 数列6, 4, 2, 0, -2, -4; 
						\item 数列 1, 1, 1, 1, 1; 
						\item 数列 -3, -2, -1, 1, 2, 3. 
					\end{enumerate}
				\vspace{10pt}

					当公差$d$取不同值时数列具有的特点: 
						\begin{description}
							\item[$d>0$时,] $\{a_n\}$为\underline{\hspace{30pt}}数列; 
							\item[$d<0$时,] $\{a_n\}$为\underline{\hspace{30pt}}数列; 
							\item[$d=0$时,] $\{a_n\}$为\underline{\hspace{30pt}}数列. 
						\end{description}


			\textbf{等差数列:}一般地,如果一个数列从\underline{\hspace{40pt}},每一项与它的前一项的差等于\underline{\hspace{40pt}},那么这个数列就叫做等差数列. 这个常数叫做等差数列的\underline{\hspace{30pt}},通常用字母\underline{\hspace{15pt}}表示。
			\vspace{20pt}

			用递推公式描述等差数列的定义:
			\underline{\hspace{60pt}} 或 \underline{\hspace{60pt}}. 



			\subsubsection{等差数列的通项公式} 

				如果一个数列$a_1, a_2, a_3, \ldots, a_n, \ldots $是等差数列,它的公差是$d$, 则它的通项公式为\underline{\hspace{70pt}}. 



		\subsection{例题讲解} % (fold)
		\label{sub:例题讲解}

			\textbf{题型一: 求通项$a_n$}

				例1.(1) $a_1=1, d=2, n=10$, 求$a_{10}=?$ $a_n=?$ 
				\vspace{30pt}

				(2) 已知等差数列$8, 5, 2, \ldots$ 求$a_n$及$a_{20}$. 
				\vspace{40pt}
					
				练习1: 已知等差数列$3, 7, 11, \ldots$, 则\\ 
				$a_n=\underline{\hspace{30pt}}$ \hspace{30pt} 
				$a_4=\underline{\hspace{30pt}}$ \hspace{30pt} 
				$a_{10}=\underline{\hspace{30pt}}$
	
	

			\textbf{题型二: 求首项$a_1$}

				例2. 已知等差数列$\{a_n\}$中, $a_{20}=-49, d=-3$, 求首项$a_1$.
				\vspace{40pt}
					

				练习2: $a_4=15, d=3, 则a_1=?$
				\vspace{20pt}
	
	

			\textbf{题型三: 求项数$n$}

				例3. 判断$-400$是不是等差数列$-5, -9, -13, \ldots$的项?如果是, 是第几项? 
				\vspace{40pt}

				练习3: $100$是不是等差数列$2, 9, 16, \ldots$的项? 如果是, 是第几项? 如果不是, 说明理由. 
				\vspace{20pt}
	
	

			\textbf{题型四: 列方程组求基本量}

				例4. 在等差数列中,已知$a_5=10, a_{12}=31$, 求首项$a_1$与公差$d$. 
				\vspace{40pt}



				【拓展训练】 例5. 数列$\{a_n\}$ 是等差数列,$a_p=q, a_q=p$ ($p,q \in \mathbf{N}^*$, 且$p\neq q$), 求$a_{p+q}$. 
				\vspace{40pt}



		% subsection 例题分析 (end)


		\subsection{课堂小结}
			通过这节课的学习,需要大家掌握的知识有以下四点:\\
			\textbf{一个定义} \underline{\hspace{60pt}} \hspace{30pt} \textbf{一个方法} \underline{\hspace{60pt}}\\
			\textbf{一个公式} \underline{\hspace{60pt}} \hspace{30pt} \textbf{一个思想} \underline{\hspace{60pt}}


		\section{课后作业}

			《教材》 40页 \hspace{10pt}\textbf{习题2.2 } \hspace{10pt}  A组 1,3,4题  \hspace{10pt}  B组 2题

			探究性作业:在自己身边找出一个等差数列的例子,找出它的首项、公差. 

\end{document}