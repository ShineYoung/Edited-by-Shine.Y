\documentclass[10pt,a4paper]{article}

\usepackage[UTF8]{ctex}
\usepackage{geometry}
\usepackage{amsmath}
\usepackage{xcolor}
\usepackage{color}

\geometry{a4paper,scale=0.8}

\begin{document}

\centerline{\Large{\textbf{2.2 等差数列}}}
\rightline{备课人:杨习}

	\section{教学重难点 及 三维目标}

		\begin{description}

			\item[重点] 理解等差数列的概念, 探索并掌握等差数列的通项公式; 

			\item[难点] 等差数列通项公式推导. 

		\end{description}

%   \section{三维目标}

  		\subsection{知识与技能}
			\begin{enumerate}
				\item 通过实例,把握等差数列的特点,理解等差数列的概念,能根据定义判断一个数列是等差数列; 
				\item 探索并掌握等差数列的通项公式,及通项公式的简单应用. 
			\end{enumerate}

		\subsection{情感态度与价值观}
			通过等差数列概念的归纳概括,培养学生的观察、分析资料的能力,积极思维,追求新知的创新意识. 

		\subsection{过程与方法}
			\begin{enumerate}

				\item 合作探究, 让学生对生活中实际问题分析, 引导学生通过观察, 推导, 归纳抽象出等差数列的概念; 

				\item 通过探索, 推导等差数列的通项公式, 并解决相应的问题; 

				\item 让学生用所学的知识解决相关的问题,归纳整理本节所学知识. 

			\end{enumerate}



	\section{教学过程}

		\subsection{复习引入} % (fold)
		\label{sub:复习引入}
		
		% subsection 复习引入 (end)
			
			【复习旧知】

			\begin{itemize}
				\item 数列的概念
				\item 数列的分类方式
				\item 数列的通项公式
			\end{itemize}
			
			【生活中的数列】
			
			姚明刚进NBA时一周里每天需要训练的罚球个数:第一天 6000个,第二天 6500个,第三天 7000个,第四天 7500,第五天 8000,第六天 8500,第七天 9000. 

			女式运动鞋的尺码数:22, 22.5, 23, 23.5, 24, 24.5, 25. 

			哈雷彗星的回归年份:1682年,1758年,1834年,1910年,1986年,2062年. 
			\vspace{10pt}

			【小组探究,提问环节】\\
			观察这些数列有什么样的共同特点? 
			\vspace{10pt}

			共同特征:从第二项起,每一项与它前面一项的差等于同一个常数(即等差);(注:每相邻两项的差相等——应指明作差的顺序是后项减前项),我们给具有这种特征的数列一个名字——等差数列。(\emph{板书:本节课标题 2.2 等差数列})

		\subsection{探究新知} % (fold)
		\label{sub:探究新知}
		
		% subsection 探究新知 (end)
			\subsubsection{等差数列的定义}
			\textbf{等差数列:}一般地,如果一个数列从第二项起,每一项与它前一项的差等于同一个常数,这个数列就叫做等差数列,这个常数就叫做等差数列的公差(常用小写字母“$d$”表示)。

			【提问环节】\\
			练习:判断下列数列中哪些是等差数列,哪些不是?如果是,写出公差$d$, 如果不是,说明理由。 

			\begin{enumerate}
				\item 数列4, 7, 10, 13, 16, $\ldots$; \hspace{10pt} 公差是3
				\item 数列6, 4, 2, 0, -2, -4; \hspace{10pt} 公差是-2
				\item 数列 1, 1, 1, 1, 1; \hspace{10pt} 公差是0
				\item 数列 -3, -2, -1, 1, 2, 3. \hspace{10pt} 不是
			\end{enumerate}

			判断一个数列是不是等差数列,主要是根据定义来判断每一项(从第2项起)与它的前一项的差是不是同一个常数,而且公差可以是正数,负数,也可以为0.

			当公差$d$取不同值时数列具有的特点: 

			\begin{description}
				\item[$d>0$时,] $\{a_n\}$为递增数列; 
				\item[$d<0$时,] $\{a_n\}$为递减数列; 
				\item[$d=0$时,] $\{a_n\}$为常数列. 
			\end{description}

			复述一遍等差数列的定义。

			【提问环节】\\
			请大家思考一下是否可以用递推公式来描述等差数列的定义呢? (\emph{板书 1.定义})

			% $a_n-a_{n-1}=d$($d$是常数, $ n\geq 2, n\in \mathbf{N_+} $) 或 $a_{n+1}-a_{n}=d$($d$是常数, $ n\in \mathbf{N_+} $)

			【思考】\\
			等差数列的递推公式是这样的,那等差数列是否有通项公式呢?它的通项公式又是什么样子的呢?


			\subsubsection{等差数列的通项公式} 

				如果一个数列$a_1, a_2, a_3, \ldots, a_n, \ldots $是等差数列,它的公差是$d$: 

				【提问环节】\\
				根据定义找一下$a_2$和$a_1$之间有什么样的关系?

				\textbf{不完全归纳法:}\\
				$a_2=a_1+d$\\ 
				$a_3=a_2+d=a_1+2d$\\
				$a_4=a_3+d=a_1+3d$\\
				$a_5=a_4+d=a_1+4d$\\
				$\hspace{2pt}\vdots \hspace{26pt} \vdots$\\
				$a_n=a_1+(n-1)d$

				【思考】\\
				是否还有其他办法来推导等差数列的通项公式? (\emph{板书: 2.通项公式及累加法})
				\newpage

				\textbf{累加法:}\\
				根据定义有:\\
				$a_2-a_1=d$\\ 
				$a_3-a_2=d$\\ 
				$a_4-a_3=d$\\ 
				$a_5-a_4=d$\\ 
				$\hspace{2pt}\vdots \hspace{26pt} \vdots$\\
				$a_n-a_{n-1}=d$

				等号两边分别相加得:$a_n-a_1=(n-1)d$,所以$a_n=a_1+(n-1)d$. 


				【提问环节】\\
				这个式子里面涉及到的独立的量有哪些?

				数学思想——“知三求一”的方程思想求解等差数列相关问题。



		\subsection{例题分析} % (fold)
		\label{sub:例题分析}

			\textbf{题型一: 求通项$a_n$}

				例1.(1) $a_1=1, d=2, n=10$, 求$a_{10}=?$\\ 
				\textcolor[rgb]{0.5,0.5,0.5}{解: $a_{10}=a_1+9d  =19$}

				进一步提问:$a_n=?$\\
				\textcolor[rgb]{0.5,0.5,0.5}{解:$a_n=1+(n-1)\cdot 2=2n-1$}

				(2) 已知等差数列$8, 5, 2, \ldots$ 求$a_n$及$a_{20}$.\\ 
				\textcolor[rgb]{0.5,0.5,0.5}{解: 由题知, \\ $a_1=8$\\ 
				$d=5-8=-3$\\ 
				因此$a_n=8+(n-1)\cdot(-3)=-3n+11$\\ 
				故$a_{20}=-49$}
					

				练习1: 已知等差数列$3, 7, 11, \ldots$, 则\\ 
				$a_n=\textcolor[rgb]{0.5,0.5,0.5}{\underline{4n-1}}$ \hspace{30pt} 
				$a_4=\textcolor[rgb]{0.5,0.5,0.5}{\underline{15}}$ \hspace{30pt} 
				$a_{10}=\textcolor[rgb]{0.5,0.5,0.5}{\underline{39}}$	
	

			\textbf{题型二: 求首项$a_1$}

				例2. 已知等差数列$\{a_n\}$中, $a_{20}=-49, d=-3$, 求首项$a_1$.\\
				\textcolor[rgb]{0.5,0.5,0.5}{解: 由$a_{20}=a_1+19d$ \\  
				得$a_1=8$}
					

				练习2: $a_4=15, d=3, 则a_1=?$\\
				\textcolor[rgb]{0.5,0.5,0.5}{解:$a_1=6$}
	
	

			\textbf{题型三: 求项数$n$}

				例3. 判断$-400$是不是等差数列$-5, -9, -13, \ldots$的项?如果是, 是第几项? \\ 
				\textcolor[rgb]{0.5,0.5,0.5}{解: $a_1=-5$,  $d=-4$,  $a_n=-5+(n-1)\cdot(-4)$\\ 
				假设 $-400$是该等差数列中的第$n$项, \\ 
				则 $-400=-5+(n-1)\cdot(-4)$ \\ 
				解之得, $n=\frac{399}{4} \hspace{20pt}$(不是正整数) \\ 
				所以$-400$不是这个数列的项. }
					

				练习3: $100$是不是等差数列$2, 9, 16, \ldots$的项? 如果是, 是第几项? 如果不是, 说明理由. \\ 
				\textcolor[rgb]{0.5,0.5,0.5}{解:是, 第15项}
	\newpage
	

			\textbf{题型四: 列方程组求基本量}

				例4. 在等差数列中,已知$a_5=10, a_{12}=31$, 求首项$a_1$与公差$d$. \\ 	

				\textcolor[rgb]{0.5,0.5,0.5}{解: 由题意知\\
				\[ \left\{ 
					\begin{array}{l}
						10=a_1+4d\\
						31=a_1+11d
					\end{array}
					\right. \] 	
				解方程组得:\\
				\[ \left\{ 
					\begin{array}{l}
						a_1=-2\\
						d=3
					\end{array}
					\right. \]}

				【拓展训练】 例5. 数列$\{a_n\}$ 是等差数列,$a_p=q, a_q=p$ ($p,q \in \mathbf{N}^*$, 且$p\neq q$), 求$a_{p+q}$.

		% subsection 例题分析 (end)


		\subsection{课堂小结}
		
			\begin{description}
			\item[一个定义] $a_{n+1}-a_n=d$
			\item[一个方法] 累加法
			\item[一个公式] $a_n=a_1+(n-1)d$
			\item[一个思想] 方程思想
			\end{description}


		\section{课后作业}

			《教材》 40页 \hspace{10pt}\textbf{习题2.2 } \hspace{10pt}  A组 1,3,4题  \hspace{10pt}  B组 2题

			探究性作业:在自己身边找出一个等差数列的例子,找出它的首项、公差. 

\end{document}