%!TeX Program = xelatex

\documentclass{article}

\usepackage[UTF8]{ctex}
\usepackage{amsmath}
\usepackage{amssymb}

\title{课时作业(七)  --详细分析}
\author{Created by Shine Y}

\begin{document}

\maketitle

以下是为没有时间讲到的习题专门制作的一份详细的解析,请同学们一定要认真研究我特意解析出来的题目。
\vspace{20pt}

3. 集合$M=\{x|x=1+a^2, a\in \mathbf{N}^*\}, P=\{x|x=a^2-4a+5, a\in \mathbf{N}^*\}$,则下列关系中正确的是(A.  $M\subsetneqq P$)

解析:

根据题目,我们知道 集合M 表示当$a$取值为$1,2,3,4,5 \cdots$时,$x$的所有值组成的集合,类似地,集合N 也表示当$a$取值为$1,2,3,4,5 \cdots$时,$x$的所有值组成的集合。

当我们看到$x=1+a^2$时,我们马上想到这里面存在两个正在变化的数$a$和$x$,这个式子其实就是一个二次函数式,面对一个二次函数式,当我们觉得用一个单纯的式子并不能告诉我们更多有用信息的时候,我们马上想到将它转化为一个图像,这就是“数形结合”的思想。于是我们就将$x=1+a^2$的图像画出来,对二次函数式$x=a^2-4a+5$也同样画出它的图像。我们知道二次函数式的图像是一条抛物线,我们画出来的结果如下图所示:



\end{document} 