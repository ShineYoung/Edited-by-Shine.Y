\documentclass{article}

\usepackage[UTF8]{ctex}
\usepackage{amsmath}
\usepackage{fancyhdr}
\usepackage{layout}
\usepackage{flushend, cuted}
\usepackage{mathpazo}
\usepackage{tikz}

\pagestyle{fancy}

\addtolength{\hoffset}{-1.5cm}
\addtolength{\voffset}{-2.5cm}
\addtolength{\textwidth}{2.5cm}
\addtolength{\textheight}{5cm}

%\author{Shine Young}
\newcommand*{\num}{10}

\tikzset{elegant/.style={smooth,thick,sample=50,cyan}}
\tikzset{eaxis/.style={->,>=stealth}}

\begin{document}

\thispagestyle{plain}

\centerline{\Large \textbf{二次的 \dots 的超级特训}}
\vspace{30pt}

{\large“我眼中的大自然是一个我们只能非常不完美地理解的宏伟机构,这必然使一个沉思者充满了谦卑的感觉”}\\
\rightline{\textbf{-- 阿尔伯特·爱因斯坦}}
\vspace{30pt}


一元二次类的函数、方程、不等式等在高中阶段有着很广的应用,所以掌握这种基本功将会让你事半功倍。每次一碰见涉及到这一类问题的题目,你就再也不会犯怵了。


\section{二次函数}

\begin{tabular}{c l l l}

  % after \\: \hline or \cline{col1-col2} \cline{col3-col4} ...
  一般式 & $y=ax^2+bx+c$ & $x=-\frac{b}{2a}$ & $(-\frac{b}{2a},\frac{4ac-b^2}{4a})$ \\
  顶点式 & $y=a(x-h)^2+k$ & $x=h$ & $(h,k)$ \\
  交点式 & $y=a(x-x_1)(x-x_2)$ & $x=\frac{x_1+x_2}{2}$ &  \\

\end{tabular}
\vspace{20pt}

\subsection{几点关于二次函数要特别记住的}

\begin{itemize}
  \item 在二次函数中,出现了: 对任意$t\in \mathbf{R}$, 都有 $\mathbf{f(a+t)=f(a-t)}$, 则该二次函数的对称轴为$x=a$.
  \item $x_1+x_2=-\frac{b}{a}$\\$x_1x_2=\frac{c}{a}$
  \item $(x_1-x_2)^2=(x_1+x_2)^2-4x_1x_2$
\end{itemize}

\subsection{二次函数图象}
当$a>0$时
\begin{tikzpicture}
  
  \draw[eaxis](-\num,0)--(\num,0) node[below]{$x$};
  \draw[eaxis](-\num,0)--(\num,0) node[above]{$y$};
  
%  \draw[elegant, domain=-\num:\num] plot(\x, {x^2+2*x-1})
  
  
\end{tikzpicture}


\section{二次方程}




\section{二次不等式}




\end{document} 