\documentclass[headheight=4.5cm,
			   margin=2cm,
			   titlewidth=0.6,
			   sansserif,
			   firstcolor=color1,
			   secondcolor=color2,
			   logo=myLogo.png,
			   % footband=myFootBand.png
			  ]{TelecomNancy}

\definecolor{color1}{RGB}{100, 100, 100}
\definecolor{color2}{RGB}{220, 0, 0}
\usepackage{amsmath}
\usepackage{amssymb}
\usepackage{pifont}
\usepackage{tikz}
\usepackage{tabularx}
\usepackage{array}
\usepackage{booktabs}
\usepackage{multirow}
\usepackage{tabu}
\usepackage{caption}
\usepackage[UTF8]{ctex}

\begin{document}

	\coursetitle{三角函数 基本知识清单}
	\courselevel{高中数学一年级}
	\courseyear{2018.03.14}
	
	\globalinstructions[填空前的提示]
	{
		 在填这些空之前,希望大家能借助这次机会,将基本的知识性概念再熟络于心,千万不要忽视基础的威力,数学本身就是一门基础的学科,但是数学的影响力却非常物所能及. 
	}

		\nextExercise[任意角的三角函数]{}

		\nextQuestion{正弦、余弦、正切函数值 在各象限的符号问题:\\
		% 请仿照$\sin \alpha$在各坐标轴以及在各象限的符号情况,写出$\cos \alpha$及$\tan \alpha$的符号情况.\\
		% \begin{minipage}{0.2\textwidth}
		% 	\begin{tikzpicture}
		% 		\draw[->] (-1.2,0) node[left]{$0$} -- (1.2,0) node[right]{$0$};
		% 		\draw[->] (0,-1.2) node[below]{$-1$} node[below=10pt]{$\sin \alpha$} -- (0,1.2) node[above]{$1$};
		% 		\draw (0,0) node[anchor=south west]{$+$} node[anchor=south east]{$+$} node[anchor=north east]{$-$} node[anchor=north west]{$-$};
		% 	\end{tikzpicture}
		% \end{minipage}
		% \begin{minipage}{0.2\textwidth}
		% 	\begin{tikzpicture}
		% 		\draw[->] (-1.2,0) -- (1.2,0);
		% 		\draw[->] (0,-1.2) node[below=10pt]{$\cos \alpha$} -- (0,1.2);
		% 	\end{tikzpicture}
		% \end{minipage}
		% \begin{minipage}{0.2\textwidth}
		% 	\begin{tikzpicture}
		% 		\draw[->] (-1.2,0) -- (1.2,0);
		% 		\draw[->] (0,-1.2) node[below=10pt]{$\tan \alpha$} -- (0,1.2);
		% 	\end{tikzpicture}
		% \end{minipage}
		\begin{center}
		% \begin{minipage}{0.5\textwidth}
			\begin{tikzpicture}
				\draw[->] (-1.2,0) -- (1.2,0);
				\draw[->] (0,-1.2) -- (0,1.2);
				\draw (0,0) node[anchor=south east]{$\sin$} node[anchor=north east]{$\tan$} node[anchor=north west]{$\cos$};
			\end{tikzpicture}
			\captionof{figure}{各象限为正值的三角函数分布情况}
		% \end{minipage}
		\end{center}

		}

		\nextQuestion{常用特殊角 三角函数值表:\\
		\begin{tabular}{c|c|c|c|c|c|c|c|c|c|c}
		\hline
		角$\alpha$ & $\phantom{1}0^\circ\phantom{1}$ & $\phantom{1}30^\circ$ & $\phantom{1}45^\circ$ & $\phantom{1}60^\circ$ & $\phantom{1}90^\circ$ & $120^\circ$ & $135^\circ$& $150^\circ$ & $180^\circ$ & $270^\circ$\\
		\hline
		角$\alpha$的弧度数 & $\quad$ & $\quad$ & $\quad$ & $\quad$ & $\quad$ & $\quad$ & $\quad$ & $\quad$ & $\quad$ & $\quad$\\
		\hline
		$\sin\alpha$ & $\quad$ & $\quad$ & $\quad$ & $\quad$ & $\quad$ & $\quad$ & $\quad$ & $\quad$ & $\quad$ & $\quad$\\
		\hline
		$\cos\alpha$ & $\quad$ & $\quad$ & $\quad$ & $\quad$ & $\quad$ & $\quad$ & $\quad$ & $\quad$ & $\quad$ & $\quad$\\
		\hline
		$\tan\alpha$ & $\quad$ & $\quad$ & $\quad$ & $\quad$ & $\quad$ & $\quad$ & $\quad$ & $\quad$ & $\quad$ & $\quad$\\
		\hline

		\end{tabular}
		}

		\nextQuestion{同角三角函数的基本关系式:\\
		$\sin^2\alpha+\cos^2\alpha=\underline{\hspace{0.5cm}} \quad \tan\alpha=\underline{\hspace{1cm}}$
		}


	\nextExercise[诱导公式]{口诀:“\textbf{奇变偶不变,符号看象限}”.\\
	\ding{172} “奇”“偶”指“$\frac{\pi}{2}$”的奇数倍和偶数倍;\\
	\ding{173} “变”与“不变”是指函数名称是否改变;\\
	\ding{174} 把$\alpha$ 当作锐角 $\longrightarrow$ 找象限 $\longrightarrow$ 判断符号.
	}

		\begin{tabularx}{0.9\textwidth}{c|X|X|X|X|X|X|X|X|X}
			\hline
			 & $2k\pi+\alpha$ & $2\pi-\alpha$ & $\phantom{2}\pi+\alpha$ & $\phantom{2}\alpha-\pi$ & $\phantom{2}-\alpha$ & $\phantom{2}\frac{\pi}{2}-\alpha$ & $\phantom{2}\frac{\pi}{2}+\alpha$ & $\frac{3\pi}{2}-\alpha$ & $\frac{3\pi}{2}+\alpha$\\
			\hline
			$\sin$ & & & & & & & & & \\
			\hline
			$\cos$ & & & & & & & & & \\
			\hline
			$\tan$ & & & & & & & & & \\
			\hline
		\end{tabularx}

	\newpage
	
	\nextExercise[和角差角 正弦、余弦、正切 公式]{}
		\nextQuestion{和差角 正余弦、正切公式}
			$\cos(\alpha+\beta)=\underline{\hspace{5cm}}$\\
			$\cos(\alpha-\beta)=\underline{\hspace{5cm}}$\\
			$\sin(\alpha+\beta)=\underline{\hspace{5cm}}$\\
			$\sin(\alpha-\beta)=\underline{\hspace{5cm}}$\\
			$\tan(\alpha+\beta)=$\\
			$\phantom{\tan(\alpha+\beta)=}\underline{\hspace{5cm}}$\\
			$\tan(\alpha-\beta)=$\\
			$\phantom{\tan(\alpha+\beta)=}\underline{\hspace{5cm}}$\\
		\nextQuestion{二倍角公式}
			$\sin2\alpha=\underline{\hspace{5cm}}$\\
			$\cos2\alpha=\underline{\hspace{5cm}}$\\
			$\phantom{\cos2\alpha}=\underline{\hspace{5cm}}$\\
			$\phantom{\cos2\alpha}=\underline{\hspace{5cm}}$\\
			$\tan2\alpha=$\\
			$\phantom{\tan2\alpha=}\underline{\hspace{5cm}}$\\
			$1+\cos\alpha=\underline{\hspace{5cm}}$\\
			$1-\cos\alpha=\underline{\hspace{5cm}}$\\
			$1+\sin\alpha=\underline{\hspace{5cm}}$\\
			$1-\sin\alpha=\underline{\hspace{5cm}}$\\
			% \vspace{17pt}
			$\sin^2\frac{\alpha}{2}=$\\
			$\phantom{\sin^2\frac{\alpha}{2}=}\underline{\hspace{5cm}}$\\
			$\cos^2\frac{\alpha}{2}=$\\
			$\phantom{\cos^2\frac{\alpha}{2}=}\underline{\hspace{5cm}}$\\
			$\tan^2\frac{\alpha}{2}=$\\
			$\phantom{\tan^2\frac{\alpha}{2}=}\underline{\hspace{5cm}}$\\
		\nextQuestion{半角公式}
			$\sin\frac{\alpha}{2}=$\\
			$\phantom{\sin\frac{\alpha}{2}=}\underline{\hspace{5cm}}$\\
			$\cos\frac{\alpha}{2}=$\\
			$\phantom{\cos\frac{\alpha}{2}=}\underline{\hspace{5cm}}$\\
			$\tan\frac{\alpha}{2}=$\\
			$\phantom{\tan\frac{\alpha}{2}=}\underline{\hspace{5cm}}$\\
			$\phantom{\tan\frac{\alpha}{2}}=$\\
			$\phantom{\tan\frac{\alpha}{2}=}\underline{\hspace{5cm}}$\\
			$\phantom{\tan\frac{\alpha}{2}}=$\\
			$\phantom{\tan\frac{\alpha}{2}=}\underline{\hspace{5cm}}$\\
		\nextQuestion{化一公式}
			$a\sin x+b\cos x=\underline{\hspace{5cm}}$


\end{document}